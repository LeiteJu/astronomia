\documentclass{article}
\usepackage[brazil]{babel}
\usepackage[utf8]{inputenc}
\usepackage{url}
\usepackage{multicol}
\usepackage[bottom=2cm,top=2cm,left=2cm,right=2cm]{geometry}

\title{Fundamentos de Astronomia\\
    \large \texttt{[AGA0215]}\\
    \large Prof Auguto Damineli\\
    \large Prof Eduardo Cypriano}

\author{Julia Leite}

\begin{document}
    
\maketitle

\tableofcontents

\newpage

\section{Astronomia e humanidade}

\subsection{Origem}

Texto mais antigo sobre astronomia encontrado na mesopotâmia (atual
Iraque) tem observações sobre o planeta Vênus, datado de 1600 a.C.

Existem gravuras mais antigas, mas que não podem ser traduzidos como 
dados, originados da China, de 2700 a.C., da Irlanda (de 3400 a.C.),
França (17 000 a.C.), entre outros....

Até animais se guiam pelos padrões dos astros, como as aves migratórias
e as plantas (fototropismo)

Antiguidade havia necessidade de entender e gerenciar os ciclos de 
tempo (estações) para saber quando plantar, estocar, etc.

Primeiro, usando a Lua para marcar tempo e estações. Só que o ciclo 
tem 29 dias e meio, então a cada 2 anos o calendário estava quase um mês
atrasado.

Depois, o Sol passou a ser usado como referência, por exemplo, quando 
ele nascia e se punha perto da estrela Aldebaran as vacas engravidaram
e a constelação (estelas perto) foi chamada de touro, simbolizando a 
primavera. Quando Regulus nascia antes do sol, começava o verão e 
foi criada a constelação de Leão. Fomalhaut indicava o outovo e Antares, o inverno. 

Essas 4 estrelas marcavam o deslocamento do sol ao longo do ano e a 
tragetória foi chamada de Zodíaco (caminho dos animais)

No Egito, as estrelas eram usadas no lugar da Lua para determinar a 
duração do ano com maior precisão, o que permitiu a formação de cidades,
excedentes na agricultura, tempo de ócio, ... 

A observação do céu também remonta a motivos religiosos, há registros 
em povos extintos da Austrália, nos guaranis, nos mesopotâmios, entre 
outros. Isso se reflete na origem da serpente como símbolo da medicina
(serpente Tiamat que traz a ideia de fazer uma viagem ritual às origens
do universo, quando não existia doença), reveillon (festa da primavera),
Natal, \dots

O céu também foi usado para transmitir ensinamentos como com a história 
do caçador Orion que invadiu a floresta de Diana, que, para castigá-lo,
mandou um escorpião picá-lo, só que o caçador é tão ligeiro que o 
escorpião nunca o alcança. Ou os boorongs que usavam as estelas Altair 
Achernar para ensinar a não praticar o incesto. 

\subsection{Astronomia moderna}

Os gregos uniram o conhecimento dos babilônios e Egito, aprimorada 
por Tales de Mileto para estruturar a astronomia sólida.

\textbf{Pitágoras} formulou que corpos celestes são redondos, seguem movimentos
circulares e a natureza se expressa por números.

\textbf{Aristóteles} usou a forma da sombra da Terra na Lua nos eclipses para
argumentar que era redonda.

\textbf{Aristarco de Samos} usou o a duração dos eclipses e a velocidade da 
Lua no céu para estimar que ela deveria ter $\frac{1}{3}$ do tamanho
da Terra e que o Sol estava distante e era maior que a Terra. Ele, 
então, formulou uma concepção heliocentrista mas que não foi muito 
aceita... 2 mil anos depois, foi defendida por Copérnico.

\textbf{Eratóstenes de Alexandria} sabendo que a Terra é redonda e medindo 
a diferença de inclinação do Sol quando se caminhava para Sul, 
calculou o diâmetro da Terra

\textbf{Hiparco de Nicéia} fez o primeiro catálogo estelar quantitativo (com
posição e brilho) e é considerado o pai da astronomia moderna, seu
modelo da Terra no centro e os astros em volta foi a base para o 
modelo de ciclos e epiciclos de \textbf{Ptolomeu}.

Com o fim da Idade Média, os árabes re-inseriram os textos gregos 
levando a um renascimento cultural, nesse contexto, \textbf{Nicolau Copérnico }
re-propôs o sistema heliocêntrico.

Mais tarde, \textbf{Johanes Kepler} utilizou as medidas de Tycho Brahe para 
descobrir a óbita elíptica de Marte.

\textbf{Galileu} observou as manchas solares e as óbitas das duas de Júpiter 
soterrando, assim, o geocentrismo. Além disso, criou a concepção de 
movimento inercial retilíneo ao invés de circular, usado, posteriormente,
por Newton.

\textbf{Isaac Newton} baseado em Galileu e Kepler para utilizou três princípios 
fundamentais para descrever o movimento de qualquer corpo.

No século XX, surgiu a teoria da relatividade de \textbf{Einstein} que mostra
o espaço como curvo, o tempo deixa de ser uniforme e buracos negros 
e lentes gravitacionais passam a ser possíveis.

A mecânica quântica desvendou o mundo sub-atômico e trouxe uma visão
dos eventos nas primeiras frações de segundo do universo.


\end{document}